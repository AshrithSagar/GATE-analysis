\documentclass{beamer}
\usetheme{Boadilla}
\usecolortheme{default}

\usepackage{tabularx, booktabs, multirow}
\usepackage[dvipsnames]{xcolor}
\usepackage{graphicx, ragged2e, microtype}
\usepackage{url, parskip, hyperref}

\hypersetup{colorlinks=true, linkcolor=blue,}
\setbeamercolor{title}{fg=black,bg=cyan}

\title[GATE]{\textbf{GATE}: Graduate Aptitude Test in Engineering}
\author[Ashrith]{\textbf{Ashrith~Sagar~Yedlapalli}}
\institute[MIT]{Manipal Institute of Technology, Manipal}
\date{\today}

\begin{document}

\section*{Titlepage}
\begin{frame}
    \titlepage
\end{frame}

\section*{Outline}
\begin{frame}{Outline}
    \tableofcontents
\end{frame}

\section{About GATE}
\begin{frame}{About GATE}{What it is}
    \begin{itemize}
        \item Conducted jointly by the IISc and the seven IITs \textbraceleft Bombay, Delhi, Guwahati, Kanpur, Kharagpur, Madras, Roorkee\textbraceright
        \item Score is used for admissions to various post-graduate education programs \textbraceleft ME, MTech, PhD\textbraceright
        \item Used for recruitments for entry-level positions to graduating engineers by several Indian Public Sector Undertakings (PSUs)
    \end{itemize}
\end{frame}

\begin{frame}{About GATE}{What it means}
    \begin{itemize}
        \item Score reflects candidate's \textbf{relative performance} level.
        \item GATE score is valid for \textbf{three years}.
        \item Date of Examination is usually in the first week of February.
        \item It's an online Computer Based Test.
    \end{itemize}
\end{frame}

\section{QP Pattern}
\begin{frame}{Pattern}{General QP pattern}
    \begin{itemize}
        \item \textbf{65 questions} carrying a total of \textbf{100 marks}
        \item \textbf{3 hours} duration
        \item QP consists of
              \begin{itemize}
                  \item General Aptitude (GA): 10 questions
                  \item Subject-specific section: 55 questions
              \end{itemize}
    \end{itemize}
\end{frame}

\end{document}
