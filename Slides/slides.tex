\documentclass[handout]{beamer}
\usetheme{Boadilla}
\usecolortheme{default}

\usepackage{tabularx, booktabs, multirow}
\usepackage[dvipsnames]{xcolor}
\usepackage{graphicx, ragged2e, microtype}
\usepackage{url, parskip, hyperref}

\hypersetup{colorlinks=true, linkcolor=blue,}
\setbeamercolor{title}{fg=black,bg=cyan}
\setbeamerfont{frametitle}{series=\bfseries}
\setbeamercolor{frametitle}{fg=black,bg=cyan!50}

\title[GATE]{\textbf{GATE}: Graduate Aptitude Test in Engineering}
\author[Ashrith]{\textbf{Ashrith~Sagar~Yedlapalli}}
\institute[MIT]{Manipal Institute of Technology, Manipal}
\date{\today}

\begin{document}

\section*{Titlepage}
\begin{frame}
    \titlepage
\end{frame}

\section*{Outline}
\begin{frame}{Outline}
    \tableofcontents
\end{frame}

\section{About GATE}
\begin{frame}{About GATE}{What it is}
    \begin{itemize}
        \item Conducted jointly by the IISc and the seven IITs \textbraceleft Bombay, Delhi, Guwahati, Kanpur, Kharagpur, Madras, Roorkee\textbraceright
        \item Score is used for admissions to various post-graduate education programs \textbraceleft ME, MTech, PhD\textbraceright
        \item Used for recruitments for entry-level positions to graduating engineers by several Indian Public Sector Undertakings (PSUs)
    \end{itemize}
\end{frame}

\begin{frame}{About GATE}{What it means}
    \begin{itemize}
        \item Score reflects candidate's \textbf{relative performance} level.
        \item GATE score is valid for \textbf{three years}.
        \item Date of Examination is usually in the first week of February.
        \item It's an online Computer Based Test.
    \end{itemize}
\end{frame}

\section{QP Pattern}
\begin{frame}{Pattern}{General QP pattern}
    \begin{itemize}
        \item \textbf{65 questions} carrying a total of \textbf{100 marks}
        \item \textbf{3 hours} duration
        \item QP consists of
              \begin{itemize}
                  \item General Aptitude (GA): 10 questions
                  \item Subject-specific section: 55 questions
              \end{itemize}
    \end{itemize}
\end{frame}

\begin{frame}{Pattern}{Distribution}
    \begin{table}[htbp]
    \centering
    \begin{tabular}{p{0.3\linewidth}p{0.6\linewidth}}
        \toprule
        Description           & Details                                                 \\
        \midrule
        Total Questions       & 65                                                      \\
        Type of Questions     & One-mark and Two-mark questions                         \\
        Question Segregation  & 10 General Aptitude questions (5 One-mark, 5 Two-mark)  \\
                              & 55 Technical questions                                  \\
        Distribution of Marks & General Aptitude: 15\%                                  \\
                              & Technical and Engineering Mathematics: 85\%             \\
        Marking Scheme        & Negative marking for wrong MCQ answers                  \\
                              & Deduction of 1/3rd original marks for wrong MCQ answers \\
                              & No negative marks for MSQs and NATs                     \\
        Partial Credit        & No partial credit for MSQs and NATs                     \\
        \bottomrule
    \end{tabular}
    \caption{GATE Examination Overview}
    \label{tab:gate-overview}
\end{table}

\end{frame}

\section{Score}
\begin{frame}{Score calculation}{What it is}
    \begin{block}{Score calculation formula}
        \begin{equation*}
            S=S_{q}+(S_{t}-S_{q}){\frac {M-M_{q}}{{\overline {M}}_{t}-M_{q}}}
        \end{equation*}
    \end{block}
\end{frame}

\begin{frame}{Post GATE scenario}{What comes next?}
    \begin{itemize}
        \item Score is valid for \textbf{three years}.
        \item Apply through \textbf{COAP} (Common Offer Acceptance Portal).
    \end{itemize}
\end{frame}

\section{COAP portal}
\begin{frame}{COAP}{Common Offer Acceptance Portal}
    \begin{itemize}
        \item Candidates who qualify GATE can apply to various IITs and IISc through COAP.
    \end{itemize}
\end{frame}

\end{document}
