\documentclass[handout]{beamer}
\usetheme{Boadilla}
\usecolortheme{default}

\usepackage{tabularx, booktabs, multirow}
\usepackage[dvipsnames]{xcolor}
\usepackage{graphicx, ragged2e, microtype}
\usepackage{url, parskip, hyperref}

\hypersetup{colorlinks=true, linkcolor=blue,}
\setbeamercolor{title}{fg=black,bg=cyan}
\setbeamerfont{frametitle}{series=\bfseries}
\setbeamercolor{frametitle}{fg=black,bg=cyan!50}

\title[GATE]{\textbf{GATE}: Graduate Aptitude Test in Engineering}
\author[Ashrith]{\textbf{Ashrith~Sagar~Yedlapalli}}
\institute[MIT]{Manipal Institute of Technology, Manipal}
\date{\today}

\begin{document}

\section*{Titlepage}
\begin{frame}
    \titlepage
\end{frame}

\section*{Outline}
\begin{frame}{Outline}
    \tableofcontents
\end{frame}

\section{About GATE}
\begin{frame}{About GATE}{What it is}
    \begin{itemize}
        \item Conducted jointly by the IISc and the seven IITs \textbraceleft Bombay, Delhi, Guwahati, Kanpur, Kharagpur, Madras, Roorkee\textbraceright
        \item Score is used for admissions to various post-graduate education programs \textbraceleft ME, MTech, PhD\textbraceright
        \item Used for recruitments for entry-level positions to graduating engineers by several Indian Public Sector Undertakings (PSUs)
    \end{itemize}
\end{frame}

\begin{frame}{About GATE}{What it means}
    \begin{itemize}
        \item Score reflects candidate's \textbf{relative performance} level.
        \item GATE score is valid for \textbf{three years}.
        \item Date of Examination is usually in the first week of February.
        \item It's an online Computer Based Test.
    \end{itemize}
\end{frame}

\section{QP Pattern}
\begin{frame}{Pattern}{General QP pattern}
    \begin{itemize}
        \item \textbf{65 questions} carrying a total of \textbf{100 marks}
        \item \textbf{3 hours} duration
        \item QP consists of
              \begin{itemize}
                  \item 5 + 20 = \textbf{25 questions} of \textbf{1 mark} each
                  \item 5 + 25 = \textbf{30 questions} of \textbf{2 marks} each
              \end{itemize}
        \item Sections:
              \begin{itemize}
                  \item \textbf{General Aptitude (GA)}: 15 marks, 10 questions
                        \begin{itemize}
                            \item 5 questions of 1 mark each + 5 questions of 2 marks each
                        \end{itemize}
                  \item \textbf{Subject-specific section}: 85 marks, 55 questions
                        \begin{itemize}
                            \item Engineering Mathematics: 13 marks
                            \item Subject Questions: 72 marks
                        \end{itemize}
              \end{itemize}
    \end{itemize}
\end{frame}

\begin{frame}{Pattern}{Negative markings}
    \begin{itemize}
        \item For \textbf{MCQs}, there is \textbf{negative marking}
              \begin{itemize}
                  \item \textbf{1 mark MCQs}: \textbf{-1/3} for a wrong answer
                  \item \textbf{2 marks MCQs}: \textbf{-2/3} for a wrong answer
              \end{itemize}
        \item For \textbf{MSQs}, there is \textbf{NO negative marking}.
              \begin{itemize}
                  \item Also, \textbf{NO partial marking}
              \end{itemize}
        \item For \textbf{NATs}, there is \textbf{NO negative marking}.
    \end{itemize}
\end{frame}

\begin{frame}{Pattern}{A good way to attempt}
    \begin{itemize}
        \item Start with \textbf{MCQs}, finish all the \emph{sure shot ones}
        \item Leave the \emph{doubtful} MCQs for later
        \item Attempt the MSQs
              \begin{itemize}
                  \item \textbf{NO Partial marking}, so don't spend too much time getting a few of the options right,
              \end{itemize}
    \end{itemize}
\end{frame}

\section{Score}
\begin{frame}{Score calculation}{How it works}
    \begin{block}{Score calculation formula}
        \begin{equation*}
            S=S_{q}+(S_{t}-S_{q}){\frac {M-M_{q}}{{\overline {M}}_{t}-M_{q}}}
        \end{equation*}
    \end{block}
\end{frame}

\begin{frame}{Section-wise}{A good way to attempt}
    \begin{itemize}
        \item \textbf{Engineering Mathematics} section is \textbf{easy} and \textbf{scoring}.
        \item Same with General Aptitude section.
        \item Spend no more than \emph{30 -- 40 minutes} on these two sections combined.
    \end{itemize}
\end{frame}

\begin{frame}{Subject-specific section}{A good way to attempt}
    \begin{itemize}
        \item Topics
    \end{itemize}
\end{frame}

\begin{frame}{Post GATE scenario}{What comes next?}
    \begin{itemize}
        \item Score is valid for \textbf{three years}.
        \item Apply through \textbf{COAP} (Common Offer Acceptance Portal).
    \end{itemize}
\end{frame}

\section{COAP portal}
\begin{frame}{COAP}{Common Offer Acceptance Portal}
    \begin{itemize}
        \item Candidates who qualify GATE can apply to various IITs and IISc through COAP.
    \end{itemize}
\end{frame}

\end{document}
